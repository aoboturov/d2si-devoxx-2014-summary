\documentclass[presentation]{beamer}

\usepackage{hyperref}

\begin{document}

\title[Devoxx 2014 summary]{
  \bf Devoxx 2014 summary\\
}
\author{{\bf Artem Oboturov}}
\date{April 24th 2014}

\begin{frame}
\titlepage
\end{frame}

\section{Universit\'e de la performance}

\begin{frame}
\frametitle{Universit\'e de la performance by OCTO}

Tools :
\begin{itemize}
\item \textbf{benerator} at \url{http://databene.org/} is a tool for test data generation
\item \textbf{gatling} at \url{http://gatling-tool.org/} - stress tool for HTTP
\item \textbf{graphite} at \url{http://graphite.wikidot.com/} for scalable realtime graphing (e.g. to vizualize performance results)
\begin{itemize}
\item \textbf{whisper} is a fixed-size database of numeric data over time
\item \textbf{carbon} is a backend for graphite
\end{itemize}
\item \textbf{metrics} \url{http://metrics.codahale.com}/ library to introduce points of performance measurement
\item \textbf{collectd} \url{http://collectd.org/} is a daemon which periodically collects system performance statistics
\end{itemize}

\end{frame}

\section{Microbenchmarking avec JMH}

\begin{frame}
\frametitle{Microbenchmarking avec JMH by OCTO}

\begin{itemize}
\item JMH is a Java harness for building, running, and analysing nano/micro/milli/macro benchmarks written in Java and other languages targetting the JVM
\item Opensource and a part of OpenJDK project :
\url{http://openjdk.java.net/projects/code-tools/jmh/}
\url{http://hg.openjdk.java.net/code-tools/jmh/}
\item Start learning by using samples:
\url{http://hg.openjdk.java.net/code-tools/jmh/file/tip/jmh-samples/src/main/java/org/openjdk/jmh/samples/}
\item Annotation-driven framework, generates code from the annotated sources, and bench-marks the generated version (which would better reflect a real execution)
\end{itemize}

\end{frame}

\section{Mesurer directement depuis le CPU: les compteurs de performance}

\begin{frame}
\frametitle{Mesurer directement depuis le CPU: les compteurs de performance by Ullink}

\begin{itemize}
\item Overseer is a Java framework that makes it possible \textbf{on Linux systems} by simplifying access to real-time measurement of low-level data such as Hardware Performance Counters (HPCs), IPMI sensors, and Java VM internal events. Overseer supports functionalities such as HPC-management, process/thread affinity settings, hardware topology identification, as well as power-consumption and temperature monitoring.
\item OverHpc relies on \textbf{libpfm41} for the management of HPCs.
\item Hardware topology information is gathered through the \textbf{libhwloc} library.
\item More data are acquired from IPMI-compatible sensors with the Intelligent Platform Management Interface3, a standardized interface used by system administrators to manage computer systems and monitor their operations.
\end{itemize}

\end{frame}

\section{Ansible in action - le provisionning au bon niveau d'abstraction}

\begin{frame}
\frametitle{Ansible in action - le provisionning au bon niveau d'abstraction}

\begin{itemize}
\item Ansible is an IT automation tool. It can configure systems, deploy software, and orchestrate more advanced IT tasks such as continuous deployments or zero downtime rolling updates.
\url{http://docs.ansible.com/}
\item {[Just another one]} DevOps util
\item Reusable playbooks : Ansible Galaxy
\url{https://galaxy.ansible.com/}
\end{itemize}

\end{frame}

\section{Deux ann\'ees de Continuous Delivery au pays des traders}

\begin{frame}
\frametitle{Deux ann\'ees de Continuous Delivery au pays des traders by GLE:FP}

\begin{itemize}
\item Maven plugins for deployment and versioning management:
\begin{itemize}
\item \textbf{Deployit} used to deploy a Deployment Package to an single environment
\url{http://tech.xebialabs.com/deployit-maven-plugin/}
\item \textbf{Build-number} is designed to get a unique build number for each time you build your project
\url{http://mojo.codehaus.org/buildnumber-maven-plugin/}
\item \textbf{Versions} is used when you want to manage the versions of artifacts in a project's POM
\url{http://mojo.codehaus.org/versions-maven-plugin/}
\end{itemize}
\item Deployment MUST be automated
\item Test are better in ISO-production environment
\item Release should become a "Non-event"
\end{itemize}

\end{frame}

\section{Am\'elioration de build maven}

\begin{frame}
\frametitle{Am\'elioration de build maven by Courtanet}

\begin{itemize}
\item \textbf{Inifinitest} : each time a change is made on the source code, all the tests that might fail because of those changes, are run be an IDE plugin
\url{http://infinitest.github.io/}
\item \textbf{Moreunit} another unit test framework
\url{http://moreunit.sourceforge.net/}
\item \textbf{Maven timeline plugin}
\url{https://github.com/dgageot/maven-timeline}
\end{itemize}

\end{frame}

\section{Les Applications R\'eactives : un nouveau paradigme pour lever les d\'efis de l'\'economie num\'erique}

\begin{frame}
\frametitle{Les Applications R\'eactives : un nouveau paradigme pour lever les d\'efis de l'\'economie num\'erique by InTech}

\begin{itemize}
\item \textbf{Reactive manifesto} by Typesafe (Scala, Akka, etc)
\url{http://www.reactivemanifesto.org/}
\item Look into Scala Promise API and Java 8 CompletableFuture API
\end{itemize}

\end{frame}

\section{IntelliJ IDEA tips and tricks by Jetbrains}

\begin{frame}
\frametitle{IntelliJ IDEA tips and tricks by Jetbrains}

\begin{itemize}
\item Use keyboard only
\item To learn how to use only the keyboard, try to use the \textbf{Key promoter} plugin which shows to user how one can easily make the same action using only keyboard (menus and toolbar button mouse clicks initiates shortcut display).
\item Print the IntelliJ keymap for your OS and use it
\item Use structural search
\item Multi-cursor edit
\end{itemize}

\end{frame}

\section{Building a Real-Time Risk Analysis System in Java}

\begin{frame}
\frametitle{Building a Real-Time Risk Analysis System in Java by GLE:FP (1)}

Maven best practices proposed :
\begin{itemize}
\item source: \textbf{javaformatter} + \textbf{jrx}
\url{https://code.google.com/p/maven-java-formatter-plugin/}
\url{https://maven.apache.org/plugins/maven-jxr-plugin/}
\item packaging: \textbf{appassembler} + \textbf{war} + \textbf{assembly}
\url{http://mojo.codehaus.org/appassembler/appassembler-maven-plugin/}
\item release: \textbf{release} + \textbf{changes} + \textbf{buildnumber}
\url{https://maven.apache.org/plugins/maven-changes-plugin/}
\url{http://maven.apache.org/maven-release/maven-release-plugin/}
\end{itemize}

\end{frame}

\begin{frame}
\frametitle{Building a Real-Time Risk Analysis System in Java by GLE:FP (2)}

Maven best practices proposed :
\begin{itemize}
\item documentation : \textbf{site markdown} + \textbf{sample/snippet} + \textbf{linkcheck} + \textbf{pdf} + \textbf{umlgraph}
\item dependency: \textbf{version} + \textbf{overview} + \textbf{enforcer-plugin} + \textbf{dependency-tree} 
\item test : \textbf{pitest} + \textbf{failsafe} + \textbf{jacoco} + \textbf{checker}
\end{itemize}

Exception monitoring in logs \textbf{(The Elasticsearch ELK Stack)}
\begin{itemize}
\item \textbf{logstash} is a tool for managing events and logs
\url{http://logstash.net/}
\item \textbf{elastic search}
\url{http://www.elasticsearch.org/}
\item \textbf{kibana} - the RT data visualization dashboard
\url{http://www.elasticsearch.org/overview/kibana/}
\end{itemize}

\end{frame}

\section{BOF BrownBagLunch France}

\begin{frame}
\frametitle{BOF BrownBagLunch France}

The concept (from Wikipedia):
a \textbf{brown bag seminar}, session or lunch is generally a training or information session during a lunch break. The term "brown bag" refers to the packed lunch meals that are either brought along by the attendees or provided by the host. In the USA, these are often packed in brown paper bags. Brown bag seminars will normally run for one or two hours.
\url{http://en.wikipedia.org/wiki/Brown_bag_seminar}

\vskip5pt
The site:
\url{http://www.brownbaglunch.fr/}

\vskip5pt
Specialists list:
\url{http://www.brownbaglunch.fr/baggers.html}

\end{frame}

\end{document}

